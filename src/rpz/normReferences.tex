\section*{Нормативные ссылки}

В дипломном проектировании использованы следующие стандарты:
\begin{itemize}
\item ГОСТ 2.105-95 -- «ЕСКД. Общие требования к текстовым документам».
\item ГОСТ 7.1-2003  -- «СИБИД. Библиографическая запись. Библиографическое описание. Общие требования и правила составления».
\item ГОСТ 7.12-93 -- «СИБИД. Библиографическая запись. Сокращение слов на русском языке».
\item ГОСТ 7.32-2001 -- «СИБИД. Отчет о научно-исследовательской работе. Структура и правила оформления».
\item ГОСТ 19.701-90 -- «ЕСКД. Схемы алгоритмов, программ, данных и систем. Условные обозначения и правила выполнения».
\item СанПиН 2.2.1/2.1.1.1278-03 -- «Гигиенические требования к естественному, искусственному и совмещенному освещению жилых и общественных зданий».
\item СанПиН 2.2.2/2.4.1340-03 -- «Гигиенические требования к видеодисплейным терминалам, персональным электронно-вычислительным машинам и организации работы».
\item СНиП 2.04.05-86  --  «Отопление, вентиляция и кондиционирование».
\end{itemize}

В расчетно-пояснительной записке имеются ссылки на следующие стандарты:
\begin{itemize}
\item ГОСТ 12.1.002-84 -- «ССБТ. Электрические поля промышленной частоты. Допустимые уровни напряженности и требования к проведению контроля на рабочих местах».
\item ГОСТ 12.1.003-83 -- «ССБТ. Шум. Общие требования безопасности».
\item ГОСТ 12.1.004-91 -- «ССБТ. Пожарная безопасность. Общие требования».
\item ГОСТ 12.1.005-88 -- «ССБТ. Общие санитарно-гигиенические требования к воздуху рабочей зоны».
\item ГОСТ 12.1.006-84 -- «ССБТ. Электромагнитные поля радиочастот. Допустимые уровни на рабочих местах и требования к проведению контроля».
\item ГОСТ 12.2.007.3-75 -- «ССБТ. Электротехнические устройства на напряжение свыше 1000 В. Требования безопасности».
\item ГОСТ 12.2.007.4-75 -- «ССБТ. Шкафы комплектных распределительных устройств и комплектных трансформаторных подстанций, камеры сборные одностороннего обслуживания, ячейки герметизированных элегазовых распределительных устройств».
\item ГОСТ 12.1.010-76 -- «ССБТ. Взрывобезопасность. Общие требования».
\item ГОСТ 12.1.018-93 -- «ССБТ. Пожаровзрывобезопасность статического электричества. Общие требования».
\item ГОСТ 12.1.038-82 -- «ССБТ. Электробезопасность. Предельно допустимые значения напряжений прикосновения и токов».
\item ГОСТ 12.1.045-84 -- «ССБТ. Электростатические поля. Допустимые уровни на рабочих местах и требования к проведению контроля».
\item ГОСТ 12.4.124-83 -- «ССБТ. Средства защиты от статического электричества. Общие технические требования».
\item СанПиН 2.2.4.548-96 -- «Гигиенические требования к микроклимату производственных помещений».
\item СНиП 23-05-95 -- «Естественное и искусственное освещение».
\end{itemize}