\section{Технологическая часть}

\subsection{Общее описание программного комплекса}
\subsubsection{Функциональное назначение}

Подсистема предназначена для автоматизации следующих процессов:
\begin{itemize}
\item Геотегирование новостных данных
\item Предоставление инструментария для создания сценариев и аналитических заметок
\end{itemize}

Подсистема должна предоставлять веб-приложение для доступа через веб-браузер.
Подсистема должна быть проста в развёртывании.

\subsubsection{Средства технического обеспечения}

Для функционирования системы, включающей подсистему <<Волхв-Гео>> требуются как минимум следующие технические средства с нижеуказанными минимальными характеристиками:
\paragraph*{со стороны сервера:} \hfill


\begin{itemize}
\item персональная электронная вычислительная машина (ПЭВМ) типа IBM PC в
конфигурации:

\begin{itemize}
\item 2-ух ядерный процессор Intel Xeon Quad-Core (2.13GHz);
\item оперативная память не менее 4 Гбайт с возможностью расширения до 64 Гб;
\item жесткий диск с не менее, чем 100 Гбайт свободного места с возможностью горячей замены;
\item сетевой адаптер любого типа, обеспечивающий взаимодействие по локальной сети.
\end{itemize}
\end{itemize}

\paragraph*{со стороны клиента:} \hfill

\begin{itemize}
\item персональная электронная вычислительная машина (ПЭВМ) типа IBM PC в
конфигурации:

\begin{itemize}
\item процессор семейства не ниже Intel Pentium Dual-Core;
\item оперативная память не менее 2 Гбайт;
\item жесткий диск с не менее, чем 10 Гбайт свободного места;
\item сетевой адаптер любого типа, обеспечивающий взаимодействие по локальной сети.
\end{itemize}
\item клавиатура и мышь
\item цветной монитор
\end{itemize}


\subsubsection{Необходимое программное обеспечение}

Для функционирования программного комплекса необходимо обеспечить следующие программные средства: 

\paragraph*{со стороны сервера:} \hfill

\begin{itemize}
\item ОС семейства Linux, например Astra Linux 1.4 SE
\item PostgreSQL
\end{itemize}

\paragraph*{со стороны клиента:} \hfill

\begin{itemize}
\item Любая современная ОС
\item Современный веб-браузер, поддерживающий язык JavaScript и спецификацию HTML 4.0 (Mozilla Firefox 4.0, Google Chrome 10 и т.п.).
\end{itemize}

\clearpage
\subsection{Структура программы.}

Модули разрабатываемой подсистемы упаковываются в исполняемые файлы, которые содержат скомпилированный исходный код системы вместе с серверным приложением и предназначены для разворачивания на сервере.
Файлы подложки карты и контуров стран и провинций предоставляются в виде архива tar.gz и нуждаются в распаковке на сервере.

\begin{table}[h!]
\centering
\caption{Модули и файлы подсистемы}
\label{table:moduleFiles}
\begin{tabular}{L{6cm}|C{10cm}}
\multicolumn{1}{C{6cm}|}{Наименование компонента} & 
\multicolumn{1}{C{10cm}}{Файлы} \\
\hline\hline

Модуль геотегирования & Исполняемый файл geotagger, конфигурационный файл geotagger.conf \\ \hline
Модуль прогнозирования & Исполняемый файл predictor, конфигурационный файл predictor.conf \\ \hline
Модуль клиентского приложения & Исполняемый файл volchv-geo, конфигурационный файл volchv-geo.conf \\ \hline
Архив с файлами карты, стран и провинций& geo-contents.tar.gz \\ \hline

\end{tabular}
\end{table}

\clearpage
\subsection{Установка и запуск приложения}

Для настройки и запуска подсистемы необходимо выполнить следующие действия:

\begin{enumerate}
\item Установить и запустить PostgreSQL 
\item Задать роли и права доступа к базе для подсистемы
\item Создать базу
\item Загрузить исполняемый файл приложения и конфигурационные файлы подсистемы на сервер 
\item Добавить исполняемый файл в список приложений, запускаемых при загрузке ОС 
\item Скопировать статичные файлы приложения на сервер
\item Записать в конфигурационных файлах приложения данные для соединения с СУБД, путь до папки с статичными файлами и порт приложения
\item Запустить приложение
\item Проверить работоспособность системы и приступить к её эксплуатации
\end{enumerate}