\subsection{Выбор методики прогнозирования}

Необходимо выбрать методику прогнозирования для использования в подсистеме.

Варианты:
\begin{itemize}
\item Эволюционный алгоритм
\item Полиноминальная регрессия
\item Скользящее среднее
\item Комбинированный метод
\end{itemize}

Под прогнозом комбинированного метода понимается прогноз, полученный методом средневзвешенной суммы прогнозов первых трёх методик.


Критерии выбора:
\begin{itemize}
\item Быстродействие
\item Точность
\item Стабильность
\item Настраиваемость
\end{itemize}

Оценка по критериям производится путём присуждения баллов в соответствии со шкалой, представленной в таблице~\ref{table:criteria}.

Критериям соответствуют веса: 


\begin{table}[h!]
\centering
\caption{Критерии качества и их весовые коэффициенты}
\label{table:selectWeights}
\begin{tabular}{L{10cm}|C{3cm}}
\multicolumn{1}{C{10cm}|}{Критерий} & 
\multicolumn{1}{C{3cm}}{$\alpha$} \\
\hline\hline

Быстродействие & 0.2 \\
Точность & 0.4 \\
Стабильность & 0.2 \\
Настраиваемость & 0.2 \\

\end{tabular}
\end{table}

\clearpage
Произведём качественную оценку вариантов

\begin{table}[h!]
\centering
\caption{Качественная оценка}
\begin{tabular}{L{4cm}|L{2.5cm}|L{2.5cm}|L{3cm}|L{2.5cm}}
\multicolumn{1}{C{4cm}|}{Критерий} & 
\multicolumn{1}{C{2.5cm}|}{Эволюционный алгоритм} & 
\multicolumn{1}{C{2.5cm}|}{Полиноминальная регрессия} &
\multicolumn{1}{C{3cm}|}{Скользящее среднее} &
\multicolumn{1}{C{2.5cm}}{Комбинированный метод} \\
\hline\hline

Быстродействие & Удовлетворительно & Хорошо & Хорошо & Плохо \\ \hline
Точность & Хорошо & Удовлетворительно & Удовлетворительно & Отлично \\ \hline
Стабильность & Удовлетворительно & Хорошо & Отлично & Хорошо \\ \hline
Настраиваемость & Хорошо & Плохо & Плохо & Отлично \\ \hline

\end{tabular}
\end{table}

Переведём в количественную

\begin{table}[h!]
\centering
\caption{Количественная оценка}
\begin{tabular}{L{4cm}|L{2.5cm}|L{2.5cm}|L{3cm}|L{2.5cm}}
\multicolumn{1}{C{4cm}|}{Критерий} & 
\multicolumn{1}{C{2.5cm}|}{Эволюционный алгоритм} & 
\multicolumn{1}{C{2.5cm}|}{Полиноминальная регрессия} &
\multicolumn{1}{C{3cm}|}{Скользящее среднее} &
\multicolumn{1}{C{2.5cm}}{Комбинированный метод} \\
\hline\hline

Быстродействие  & 3  & 4  & 4  & 1 \\ \hline
Точность        & 4  & 3  & 3  & 5 \\ \hline
Стабильность    & 3  & 4  & 5  & 4 \\ \hline
Настраиваемость & 4  & 1  & 1  & 5 \\ \hline
Итого           & 14 & 12 & 13 & 15 \\ \hline

\end{tabular}
\end{table}

Произведём оценку с учётом весовых коэффициентов
\clearpage
\begin{table}[h!]
\centering
\caption{Оценка с учётом веса}
\begin{tabular}{L{4cm}|L{1cm}|L{2.5cm}|L{2.5cm}|L{3cm}|L{2.5cm}}
\multicolumn{1}{C{4cm}|}{Критерий} & 
\multicolumn{1}{C{1cm}|}{$\alpha$} &
\multicolumn{1}{C{2.5cm}|}{Эволюционный алгоритм} & 
\multicolumn{1}{C{2.5cm}|}{Полиноминальная регрессия} &
\multicolumn{1}{C{3cm}|}{Скользящее среднее} &
\multicolumn{1}{C{2.5cm}}{Комбинированный метод} \\
\hline\hline

Быстродействие & 0.2 & 3 & 4 & 4 & 1 \\ \hline
Точность & 0.4 & 4 & 3 & 3 & 5 \\ \hline
Стабильность & 0.2 & 3 & 4 & 5 & 4 \\ \hline
Настраиваемость & 0.2 & 4 & 1 & 1 & 5 \\ \hline
Итого           & 1 & 3.6 & 3 & 3.2 & 4 \\ \hline


\end{tabular}
\end{table}

Согласно сравнению следует предпочесть комбинированную методику.