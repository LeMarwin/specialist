\documentclass[russian,utf8,emptystyle]{eskdtext}

\usepackage[T2A,T1]{fontenc}

\newcommand{\No}{\textnumero} % костыль для фикса ошибки

\ESKDdepartment{Федеральное государственное бюджетное образовательное учреждение высшего профессионального образования}
\ESKDcompany{Московский государственный технический университет им. Н. Э. Баумана}
\ESKDclassCode{23 0102}
\ESKDtitle{АИС отслеживания и прогнозирования новостных потоков «Волхв»}
\ESKDdocName{Расчетно-пояснительная записка}
\ESKDauthor{Гуща~А.~В.}
\ESKDtitleApprovedBy{~}{~\underline{\hspace{2.5cm}}}
\ESKDtitleAgreedBy{~}{~\underline{\hspace{2.5cm}}}
\ESKDtitleDesignedBy{Студент группы ИУ5-122}{Гуща~А.~В}

\usepackage{multirow}
\usepackage{tabularx}
\usepackage{tabularx,ragged2e}
\usepackage{pdfpages}
\renewcommand\tabularxcolumn[1]{>{\Centering}p{#1}}
\newcommand\abs[1]{\left|#1\right|}

\usepackage{longtable,tabu}

\usepackage{geometry}
\geometry{footskip = 1cm}

\pagenumbering{arabic}
\pagestyle{plain}

\usepackage{setspace}

\usepackage{xcolor}
\usepackage{listings}
\lstset{
    breaklines=true,
    postbreak=\raisebox{0ex}[0ex][0ex]{\ensuremath{\color{red}\hookrightarrow\space}},
    extendedchars=\true,
    basicstyle=\small,
    inputencoding=utf8,
    numbers=left,                    
    numbersep=5pt,                  
    numberstyle=\tiny\color{mygray},
}
\renewcommand{\lstlistingname}{Листинг}
\renewcommand{\lstlistlistingname}{Листинги}

\ESKDsectAlign{section}{Center} % to capitalize russian text

\usepackage{array}
\newcolumntype{L}[1]{>{\raggedright\let\newline\\\arraybackslash\hspace{0pt}}m{#1}}
\newcolumntype{C}[1]{>{\centering\let\newline\\\arraybackslash\hspace{0pt}}m{#1}}
\newcolumntype{R}[1]{>{\raggedleft\let\newline\\\arraybackslash\hspace{0pt}}m{#1}}

\usepackage{tikz}
\usepackage{pgf-pie}

\usepackage{totcount}
\regtotcounter{figure}
\regtotcounter{table}

%\usepackage{titlesec}
\usepackage{hyperref}

\setcounter{secnumdepth}{5}
\setcounter{tocdepth}{5}

%\renewcommand*{\thesection}{\arabic{section}}

\usepackage{pdfpages}
\renewcommand\tabularxcolumn[1]{>{\Centering}p{#1}}
\newcommand\abs[1]{\left|#1\right|}

\usepackage{etoolbox}
\patchcmd{\thebibliography}{\addcontentsline{toc}{section}{\refname}}{}{}{}

\usepackage{amsfonts}

\renewcommand\bibname{Список использованных источников}
\renewcommand\refname{Список использованных источников}
\AtBeginDocument{\renewcommand{\bibname}{Список использованных источников}}
\AtBeginDocument{\renewcommand{\refname}{Список использованных источников}}

\usepackage{fancyhdr}
\usepackage{atenddvi}
\usepackage[user]{zref}

\usepackage{blindtext}

%===================
% Setting double page numbering
\pagestyle{fancy}
\renewcommand{\headrulewidth}{0pt}
\fancyhf{}
\fancyfoot[C]{\twopagenumbers}
\fancypagestyle{plain}{
  \renewcommand{\headrulewidth}{0pt}
  \fancyhf{}
  \fancyfoot[C]{\twopagenumbers}
}


\newcounter{pageaux}
\def\currentauxref{PAGEAUX1}
\newcommand{\twopagenumbers}{%
  \stepcounter{pageaux}%
  \thepage -- \thepageaux/\ref{\currentauxref}%
}
\makeatletter
\newcommand{\resetpageaux}{%
  \clearpage
  \edef\@currentlabel{\thepageaux}\label{\currentauxref}%
  \xdef\currentauxref{PAGEAUX\thepage}%
  \setcounter{pageaux}{0}}
\AtEndDvi{\edef\@currentlabel{\thepageaux}\label{\currentauxref}}
\makeatletter
%===========================================

%===========================================================================
\begin{document}


\includepdf[pages=1]{tz_title.pdf}
\setcounter{pageaux}{1}
\setcounter{page}{158}
\tableofcontents

\clearpage  
\section{Наименование}

Автоматизированная информационная система отслеживания и прогнозирования новостных потоков «Волхв».

\section{Основание для разработки}
Задание на дипломный проект, подписанное руководителем проекта Гапанюком Юрием Евгеньевичем..

\section{Исполнитель}

Исполнителем дипломного проекта является студент группы ИУ5-122, Гуща Антон Валерьевич.

\section{Цель разработки}

Целью разработки является создание автоматизированной информационной системы для анализа новостных потоков, а также прогнозирования их активности. АИС «Волхв» предназначена для аналитиков и исследователей в области «big data» и предоставляет основу для более сложных систем анализа событий.

\section{Содержание работы}

\subsection{Задачи, подлежащие решению}
\begin{itemize}
\item Исследование предметной области, определение функциональных задач;
\item Разработка инфологической и даталогической моделей;
\item Разработка алгоритмов основных функций программы;
\item Разработка структуры программы;
\item Программная реализация;
\item Тестирование программы;
\item Разработка конструкторской и эксплуатационной документации к программе.
\end{itemize}

\subsection{Общие требования}

\begin{itemize}
\item Программное изделие должно обеспечивать хранение данных в СУБД.
\item СУБД этой системы должна обеспечивать целостность данных.
\item Программное изделие должно обеспечивать возможность редактирования данных.
\item Программное изделие должно обеспечивать возможность просмотра данных.
\end{itemize}

\subsection{Функциональные требования}
В системе должны быть реализованы следующие функции.

\subsubsection{Удаленный доступ к системе}
Программное изделие должно обеспечивать удаленный доступ на получение информации к системе.

\subsubsection{Соединение с базой данных}
Программное изделие должно осуществлять удаленное соединение с базой данных.

\subsubsection{Поиск по сайту}
Программное изделие должно осуществлять поиск по сайту и выдавать пользователям результатов поиска.

\subsubsection{Ввод с клавиатуры}
Данные, вводимые с клавиатуры должны иметь тип и формат, соответствующий типу и формату полей записи.

\subsubsection{Добавление информации в базу данных}
Программное изделие должно осуществлять добавление новой записи в базу данных при условии, что эта запись удовлетворяет всем требованиям, налагаемым на входные данные.

\subsubsection{Удаление информации из базы данных}
Программное изделие должно осуществлять исключение выбранной пользователем записи в таблице из исходной базы данных.

\subsubsection{Редактирование информации в базе данных}

Функция должна осуществлять редактирование поля записи, выбранного пользователем. При этом при редактировании данных должны выполняться все требования, налагаемые на входные данные.

\subsection{Требования к входным данным}

Требования к входным данным налагаются в соответствие с даталогической схемой системы. 

Входными данными для сервера являются запросы к Web-серверу, запросы к серверу БД.

Входными данными для пользователя являются HTML-страницы, посылаемые ему сервером по протоколу HTTP.

\subsection{Требования к выходным данным}

Выходными данными сервера являются HTML-страницы, результаты запросов к базе данных, посылаемые пользователю по протоколу HTTP.

Выходными данными пользователя являются заполненные формы, запросы к БД, отправляемые им на сервер по протоколу HTTP.

\subsection{Требования к составу программных компонентов}

Система должна содержать следующие модули:
\begin{itemize}
\item модуль рубрикатора новостей и сохранённых запросов;
\item модуль поиска новостей и формализованных запросов;
\item модуль просмотра новости;
\item модуль управления индексом;
\item модуль интеграционного интерфейса;
\item модуль настроек системы;
\item модуль добавления документов;
\item модуль прогнозирования;
\item модуль индексирования.
\end{itemize}

\subsection{Требования к прикладным программам}

\subsubsection{Модуль рубрикатора новостей и сохранённых запросов}

Должен предоставлять пользователю интерфейс для просмотра и редактирования рубрик (категорий) новостей, а также сохранение, редактирование и удаление сохранённых запросов, которые привязываются к рубрикам. Рубрики должны образовывать древовидную структуру, листами которой являются либо рубрики, либо сохранённые запросы. Для каждой рубрики должно отображаться количество новостей, относящихся к данной рубрике. Отсутствие рубрики у сохранённого запроса или у новости должно обрабатываться и отображаться пользователю как специальная рубрика «Нерубрицированные». 

\subsubsection{Модуль поиска новостей и формализованных запросов}

Должен предоставлять пользователю интерфейс для ввода поискового запроса на естественном языке или на формализованном языке запросов. Поиск должен иметь возможность:
\begin{itemize}
\item настройки сортировки по времени новости и её релевантности запросу;
\item указания интервала времени для поиска;
\item дополнительной фильтрации по реквизитам документа;
\item поиска с условием наличия/отсутствия конкретных меток новостей.
\end{itemize}

Формализованные поисковые запросы должны иметь возможность:
\begin{itemize}
\item Поиска с учётом морфологии Русского и Английского языков;
\item Поиска с учётом максимального расстояния между ключевыми словами;
\item Поиска фиксированной фразы;
\item Поиска по конкретному реквизиту документа;
\item Поиска по ключевым словам, расположенных в конце и начале реквизита;
\item Поиска по ключевым словам с указанием приоритета для каждого из них;
\item Поиска в пределах одного предложения или параграфа;
\item Поиска с составными запросами, части которого объединены логическим оператором «ИЛИ».
\end{itemize}

\subsubsection{Модуль просмотра новости}

Должен предоставлять пользователю полную информацию о документе (новости) и его реквизитах:
\begin{itemize}
\item Заголовок новости -- краткий заголовок новости;
\item Основная часть новости -- основной массив текста с форматированием;
\item Время публикации новости -- время публикации новости, указанное в источнике;
\item Рубрика новости -- категория новости, к которой она относится;
\item Источник новости -- адрес сайта новости;
\item Идентификатор новости -- уникальный идентификатор новости, под которым она хранится в БД.
\end{itemize}

Также модуль должен предоставлять возможность редактировать реквизиты документа (за исключением идентификатора и источника) и возможность удаления новости из АИС.


\subsubsection{Модуль управления индексом}

Должен предоставлять пользователю интерфейс для совершения следующих операций над поисковым индексом:
\begin{itemize}
\item Пересоздание поискового индекса;
\item Синхронизация поискового индекса -- проверка и добавления отсутствующих документов в индексе;
\item Удаление дублей документов;
\item Перемещение документов между рубриками;
\end{itemize}

Модуль должен предоставлять возможность просмотра запущенных операций над поисковым индексом и преждевременного завершения этих задач. При включении АИС должна продолжить выполнение операций, которые были в процессе исполнения во время завершения работы АИС.

\subsubsection{Модуль интеграционного интерфейса}

Должен предоставлять пользователю интерфейс для взаимодействия с АИС других программ по протоколу HTTP, используя формат JSON. В интеграционный интерфейс должны входить следующие операции:
\begin{itemize}
\item Получение версии интерфейса;
\item Выполнение формализованных запросов;
\item Добавление, обновление и удаление документов (новостей);
\item Переиндексирование документа;
\item Добавление меток документу;
\item Создание рубрики и получения детальной информации о рубрике;
\item Получение дерева рубрикатора;
\item Перемещение документов между рубриками;
\item Создание, удаление и выполнение сохранённых запросов;
\item Получение детальной информации о сохранённом запросе;
\item Получение списка сохранённых запросов;
\item Получение и задание настроек системы;
\item Получение информации о текущей операции над поисковым индексом.
\end{itemize}

\subsubsection{Модуль настроек системы}
Должен предоставлять интерфейс для изменения следующих настроек системы:
\begin{itemize}
\item Отключение/Включение автоматического добавления документов;
\item Показ скрытых рубрик;
\item Показ отладочной информации об используемом формализованном запросе при поиске;
\item Показ всех полей в форме поиска документов;
\end{itemize}

\subsubsection{Модуль добавления документов}

Должен предоставлять пользователю интерфейс добавления документов в АИС двумя способами:
\begin{itemize}
\item Добавление через HTML форму ввода – реквизиты документа вводятся в поля формы, проходят проверку на соответствие формату входных данных и отправляются на сервер, где новый документ добавляется в БД и поисковый индекс.
\item Добавление через директорию импорта – автоматический метод добавления документов, который осуществляется через сканирование специальной директории на наличие файлов, содержащих документы в формате XML. При успешном добавлении документ перемещается в специальную директорию для добавленных документов. При неудачном добавлении документа в АИС, документ перемещается в специальную директорию для некорректных документов.
\end{itemize}

\subsubsection{Модуль прогнозирования}

Должен предоставлять возможность анализировать графики количества документов, соответствующих сохранённым запросам, по дням и проводить кратковременный прогноз этого количества документов для будущих дат. Данный модуль должен предоставлять следующую информацию пользователю:
\begin{itemize}
\item График количества документов, соответствующих сохранённому запросу, используемого для прогнозирования, по дням до текущей даты;
\item График прогнозируемого количества документов, соответствующих сохранённому запросу, используемого для прогнозирования, по дням на всём рассматриваемом интервале времени;
\item Аналитическую формулу прогноза, по которой значение количества документов, соответствующих сохранённому запросу, можно вычислить для любого дня рассматриваемого интервала времени;
\item Количественную оценку полученного прогноза;
\end{itemize}

\subsubsection{Модуль индексирования}

Должен предоставлять возможность другим модулям выполнять полнотекстовые запросы и проводить индексирование документов для осуществления полнотекстового поиска.

\subsection{Требования к архитектуре АСОИУ}

Система должна быть реализована в виде трехзвенной архитектуры <<Клиент-Сервер>> или архитектуры <<Файл-Сервер>> или в виде облачной архитектуры.

\subsection{Требования к базе данных}
Система должна обеспечивать быстродействие, целостность и безопасность базы данных. Поисковый индекс должен храниться отдельно от данных БД.

\subsection{Требования к составу технических средств}

Персональная ЭВМ архитектуры AMD x86\_64:
\begin{itemize}
\item процессор Intel i7-860 (4 ядра, 2.8 ГГц);
\item жесткий диск не менее 1 Тб;
\item оперативная память 4 Гб;
\item сетевой адаптер для подключения к ЛВС;
\item доступ в Internet по выделенной линии со скоростью не менее 256 Кбит/сек;
\end{itemize}

\subsection{Требования к составу технических средств для пользователя}

Персональная ЭВМ архитектуры AMD x86\_64 или IBM x86:
\begin{itemize}
\item процессор Intel Pentium Dual-Core;
\item жесткий диск не менее 100 Гб;
\item оперативная память 2 Гб;
\item сетевой адаптер для подключения к ЛВС.
\item Клавиатура, мышь, экран;
\end{itemize}

\subsection{Требования к программному обеспечению}

\paragraph*{Серверное ПО} \hfill

\begin{itemize}
\item Astra Linux 1.4 релиз <<Смоленск>>;
\item СУБД PostgreSQL версии 9.4;
\item Библиотека gmp версии 3 для работы с целочисленными числами произвольной точности;
\item Библиотека pcre версии 3 для работы с регулярными выражениями;
\item Клиентская библиотека от MySQL для протокола связи между сервером приложения и сервером индексации;
\item Библиотека expat для работы с XML файлами;
\end{itemize}

\paragraph*{Клиентское ПО} \hfill

\begin{itemize}
\item Любая ОС;
\item Браузер Firefox версии 46 или Chrome версии 51;
\end{itemize}

\subsection{Требования к лингвистическому обеспечению}

Пользовательский интерфейс должен быть реализован на русском языке. Ввод и вывод данных также должны осуществляться преимущественно на русском языке.

\subsection{Требования к квалификации пользователя}

Пользователь должен иметь представление о механизмах работы полнотекстового поиска. Должен иметь навык работы с GNU\Linux на уровне пользователя, что обеспечит и корректную работу с данным изделием.

\subsection{Требования к временным характеристикам}
Время реакции системы не должно быть больше 30 секунд.

\subsection{Требования к надёжности}
В связи с малым интервалом времени решения задач, задачи должны выполняться с высокой степенью надежности, близкой к единице.

\section{Порядок приёма}

Работа по курсовому проекту «Автоматизированная информационная система отслеживания и прогнозирования новостных потоков «Волхв»» принимается в установленном порядке в соответствии с предоставляемой технической документацией.

\section{Дополнительные условия}

Данное техническое задание может уточняться и изменяться в установленном порядке.

\resetpageaux

\end{document}