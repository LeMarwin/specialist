\section*{Реферат}

Данная расчетно-пояснительная записка содержит \pageref{preLastPage} (без приложения) страниц, \total{figure} иллюстраций (без приложения), \total{table} таблиц, 2 приложения, 25 использованных источников.

Ключевые слова: полнотекстовый поиск, сбор информации, формальный язык запросов, генетическое программирование, эволюционные алгоритмы, символьная регрессия, анализ временных рядов, геоинформационная система.

Данный дипломный проект посвящен разработке подсистемы анализа новостных потоков в текстовом виде, прогнозирования их активности и отображения на географической карте. Целью разработки является обработка собранной из открытых источников информации, выделение географически ограниченных новостей, анализ новостной динамики, визуализация полученных данных на географической карте, построение аналитических заметок на основе полученных данных.

Подсистема проектируется в виде веб-сервиса, предоставляющего пользователю набор экранных форм для взаимодействия с данным программных изделием, а также машинный интерфейс для взаимодействия стороннего программного обеспечения с данной подсистемой. Структуру подсистемы составляют сервер приложения, сервер СУБД, сервер индексации и терминалы пользователей. При разработке программного продукта используется язык программирования Haskell.

В результате разработки была спроектирована подсистема, отвечающая требованиям технического задания и имеющая возможности, необходимые как для всестороннего анализа новостной информации, так и для наглядного отображения её на географической карте. 

Область применения подсистемы -- аналитические центры, лаборатории анализа данных, новостные и политические агенства, органы государственной власти.

Проект является некоммерческим с малой стоимостью выполнения.