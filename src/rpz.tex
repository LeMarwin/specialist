\documentclass[russian,utf8,emptystyle]{eskdtext}

\usepackage[T2A,T1]{fontenc}

\newcommand{\No}{\textnumero} % костыль для фикса ошибки

\ESKDdepartment{Федеральное государственное бюджетное образовательное учреждение высшего профессионального образования}
\ESKDcompany{Московский государственный технический университет им. Н. Э. Баумана}
\ESKDclassCode{23 0102}
\ESKDtitle{АИС отслеживания и прогнозирования новостных потоков «Волхв»}
\ESKDdocName{Расчетно-пояснительная записка}
\ESKDauthor{Гуща~А.~В.}
\ESKDtitleApprovedBy{~}{~\underline{\hspace{2.5cm}}}
\ESKDtitleAgreedBy{~}{~\underline{\hspace{2.5cm}}}
\ESKDtitleDesignedBy{Студент группы ИУ5-122}{Гуща~А.~В}

\usepackage{multirow}
\usepackage{tabularx}
\usepackage{tabularx,ragged2e}
\usepackage{pdfpages}
\renewcommand\tabularxcolumn[1]{>{\Centering}p{#1}}
\newcommand\abs[1]{\left|#1\right|}

\usepackage{longtable,tabu}

\usepackage{geometry}
\geometry{footskip = 1cm}

\pagenumbering{arabic}
\pagestyle{plain}

\usepackage{setspace}

\usepackage{xcolor}
\usepackage{listings}
\lstset{
    breaklines=true,
    postbreak=\raisebox{0ex}[0ex][0ex]{\ensuremath{\color{red}\hookrightarrow\space}},
    extendedchars=\true,
    basicstyle=\small,
    inputencoding=utf8
}

\ESKDsectAlign{section}{Center} % to capitalize russian text

\usepackage{array}
\newcolumntype{L}[1]{>{\raggedright\let\newline\\\arraybackslash\hspace{0pt}}m{#1}}
\newcolumntype{C}[1]{>{\centering\let\newline\\\arraybackslash\hspace{0pt}}m{#1}}
\newcolumntype{R}[1]{>{\raggedleft\let\newline\\\arraybackslash\hspace{0pt}}m{#1}}

\usepackage{tikz}
\usepackage{pgf-pie}

\usepackage{totcount}
\regtotcounter{figure}
\regtotcounter{table}

%===========================================================================
\begin{document}

\clearpage
\clearpage

\section*{Реферат}

Данная расчетно-пояснительная записка содержит \pageref{LastPage} (без приложения) страниц, \total{figure} иллюстраций (без приложения), \total{table} таблиц, 2 приложения, ?? использованных источников.

Ключевые слова: полнотекстовый поиск, сбор информации, формальный язык запросов, генетическое программирование, эволюционные алгоритмы, символьная регрессия, анализ временных рядов, геоинформационная система.

Данный дипломный проект посвящен разработке подсистемы анализа новостных потоков в текстовом виде, прогнозирования их активности и отображения на географической карте. Целью разработки является обработка собранной из открытых источников информации, выделение географически ограниченных тем новостей, анализ новостной динамики, визуализация полученных данных на географической карте, построение аналитических заметок на основе полученных данных.

Подсистема проектируется в виде веб-сервиса, предоставляющего пользователю набор экранных форм для взаимодействия с данным программных изделием, а также машинный интерфейс для взаимодействия стороннего программного обеспечения с данной подсистемой. Структуру подсистемы составляют сервер приложения, сервер СУБД, сервер индексации и терминалы пользователей. При разработке программного продукта используется язык программирования Haskell.

В результате разработки была спроектирована подсистема, отвечающая требованиям технического задания и имеющая возможности, необходимые как для всестороннего анализа новостной информации, так и для наглядного отображения её на географической карте. 

Область применения подсистемы -- аналитические центры, лаборатории анализа данных, новостные и политические агенства, органы государственной власти.

Проект является некоммерческим с малой стоимостью выполнения.

\tableofcontents

\section*{Нормативные ссылки}

В дипломном проектировании использованы следующие стандарты:
\begin{itemize}
\item ГОСТ 2.105-95 -- «ЕСКД. Общие требования к текстовым документам».
\item ГОСТ 7.1-2003  -- «СИБИД. Библиографическая запись. Библиографическое описание. Общие требования и правила составления».
\item ГОСТ 7.12-93 -- «СИБИД. Библиографическая запись. Сокращение слов на русском языке».
\item ГОСТ 7.32-2001 -- «СИБИД. Отчет о научно-исследовательской работе. Структура и правила оформления».
\item ГОСТ 19.701-90 -- «ЕСКД. Схемы алгоритмов, программ, данных и систем. Условные обозначения и правила выполнения».
\item ISO 3166 -- <<Кодовые обозначения государств и зависимых территорий, а также основных административных образований внутри государств>>
\item СанПиН 2.2.1/2.1.1.1278-03 -- «Гигиенические требования к естественному, искусственному и совмещенному освещению жилых и общественных зданий».
\item СанПиН 2.2.2/2.4.1340-03 -- «Гигиенические требования к видеодисплейным терминалам, персональным электронно-вычислительным машинам и организации работы».
\item СНиП 2.04.05-86  --  «Отопление, вентиляция и кондиционирование».
\end{itemize}

В расчетно-пояснительной записке имеются ссылки на следующие стандарты:
\begin{itemize}
\item ГОСТ 12.1.002-84 -- «ССБТ. Электрические поля промышленной частоты. Допустимые уровни напряженности и требования к проведению контроля на рабочих местах».
\item ГОСТ 12.1.003-83 -- «ССБТ. Шум. Общие требования безопасности».
\item ГОСТ 12.1.004-91 -- «ССБТ. Пожарная безопасность. Общие требования».
\item ГОСТ 12.1.005-88 -- «ССБТ. Общие санитарно-гигиенические требования к воздуху рабочей зоны».
\item ГОСТ 12.1.006-84 -- «ССБТ. Электромагнитные поля радиочастот. Допустимые уровни на рабочих местах и требования к проведению контроля».
\item ГОСТ 12.2.007.3-75 -- «ССБТ. Электротехнические устройства на напряжение свыше 1000 В. Требования безопасности».
\item ГОСТ 12.2.007.4-75 -- «ССБТ. Шкафы комплектных распределительных устройств и комплектных трансформаторных подстанций, камеры сборные одностороннего обслуживания, ячейки герметизированных элегазовых распределительных устройств».
\item ГОСТ 12.1.010-76 -- «ССБТ. Взрывобезопасность. Общие требования».
\item ГОСТ 12.1.018-93 -- «ССБТ. Пожаровзрывобезопасность статического электричества. Общие требования».
\item ГОСТ 12.1.038-82 -- «ССБТ. Электробезопасность. Предельно допустимые значения напряжений прикосновения и токов».
\item ГОСТ 12.1.045-84 -- «ССБТ. Электростатические поля. Допустимые уровни на рабочих местах и требования к проведению контроля».
\item ГОСТ 12.4.124-83 -- «ССБТ. Средства защиты от статического электричества. Общие технические требования».
\item СанПиН 2.2.4.548-96 -- «Гигиенические требования к микроклимату производственных помещений».
\item СНиП 23-05-95 -- «Естественное и искусственное освещение».
\end{itemize}
\section*{Определения, обозначения и сокращения}

\begin{itemize}
\item АИС -- автоматизированная информационная система.
\item АПК -- аппаратно-программный комплекс.
\item БД -- база данных.
\item ЕСКД -- единая система конструкторской документации.
\item Управляющий класс (control) -- класс модели анализа, представляющий координацию, последовательность и управление другими объектами, часто используется для инкапсуляции управления для варианта использования.
\item Граничный класс (boundary) -- класс модели анализа, используемый для моделирования взаимодействия между системой и ее актантами, то есть пользователями и внешними системами.
\item Класс сущности (entity) -- класс модели анализа, используемый для моделирования долгоживущей, часто персистентной информации.
\item ОЗУ -- оперативное запоминающее устройство.
\item ОС -- операционная система.
\item ПК -- персональный компьютер.
\item ПО -- программное обеспечение.
\item ПЭВМ -- персональная электронно-вычислительная машина.
\item СанПиН -- санитарно-эпидемиологические правила и нормативы.
\item СИБИД -- система стандартов по информации, библиотечному и издательскому делу.
\item СНиП -- строительные нормы и правила. 
\item ССБТ -- система стандартов по информации, библиотечному и издательскому делу.
\item СУБД -- система управления базами данных.
\item ЭВМ -- электронно-вычислительная машина.
\item ЭМП -- электромагнитное поле.
\item API -- application programming interface -- набор готовых классов, процедур, функций, структур и констант, предоставляемых приложением (библиотекой, сервисом) для использования во внешних программных продуктах.
\item CPU -- central processing unit -- центральный процессор.
\item HDD -- hard disk drive -- жесткий диск.
\item SQL -- structured query language -- универсальный компьютерный язык, применяемый для создания, модификации и управления данными в реляционных базах данных. SQL основывается на исчислении кортежей. SQL является, прежде всего, информационно-логическим языком, предназначенным для описания, изменения и извлечения данных, хранимых в реляционных базах данных. 
\item UML -- unified modeling language, унифицированный язык моделирования -- язык графического описания для объектного моделирования в области разработки программного обеспечения. UML является языком широкого профиля, это открытый стандарт, использующий графические обозначения для создания абстрактной модели системы, называемой UML-моделью.
\end{itemize}
\section*{Введение}

Дипломный проект на тему <<Автоматизированная информационная система отслеживания и прогнозирования новостных потоков <<Волхв>> >> посвящен созданию системы, которая позволит накапливать информацию из открытых источников СМИ, аналитических материалов экспертов, социальных сетей, а также предоставит инструменты для всестороннего анализа накопленной информации. АИС <<Волхв>> предоставляет два способа анализа накопленной новостной информации: с помощью проблемно-ориентированного языка запросов и с помощью автоматизированного прогнозирования количества сообщений, удовлетворяющих заданному запросу.



\section{Часть 1}
\section{Часть 2}
\section{Часть 3}

%===========================================================================
% Организационно -- Экономическая часть
\include{orgecon/stages}
\include{orgecon/costs}
\include{orgecon/workers}
\include{orgecon/salary}
\include{orgecon/equip}
\include{orgecon/workplace}
\include{orgecon/overheads}
\include{orgecon/others}

%===========================================================================
% ОБЖ и Экология
\include{obzeco/analysis}
\include{obzeco/normalize}
\include{obzeco/expertise}
\include{obzeco/results}

%===========================================================================
\clearpage
\addcontentsline{toc}{chapter}{\bibname}
\bibliographystyle{utf8gost705u}  %% стилевой файл для оформления по ГОСТу
\bibliography{biblio}     %% имя библиографической базы (bib-файла) 

\end{document}